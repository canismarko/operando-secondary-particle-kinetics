\documentclass{article}
\usepackage[utf8]{inputenc}

\usepackage{graphicx}
\usepackage[version=4]{mhchem}
\usepackage{siunitx}
\usepackage{glossaries}

% Drafting
\usepackage{xcolor,soul}
\usepackage{fullpage}
\usepackage{setspace}
\usepackage{blindtext}
\doublespacing

\definecolor{mwcolor}{HTML}{8af67d}
\definecolor{vgcolor}{HTML}{c88bff} % MPL purple
\definecolor{citecolor}{HTML}{c696f0} % MPL purple

% Tools for leaving todo notes around
\usepackage[colorinlistoftodos]{todonotes}
\newcommand{\citeme}[1]{%
  \todo[color=citecolor]{%
  \ifstrempty{#1}{[citation needed]}{[cite: #1]}%
  }%
}
\newcommand{\mw}[2]{%
  \sethlcolor{mwcolor}\hl{#2}\sethlcolor{yellow}%
  \ifstrempty{#1}{}{%
    \todo[color=mwcolor]{#1 [MW]}%
  }%                  
}
\newcommand{\vg}[2]{%
  \sethlcolor{vgcolor}\hl{#2}\sethlcolor{yellow}%
  \ifstrempty{#1}{}{%
    \todo[color=vgcolor]{#1 [VG]}%
  }%                  
}

\usepackage{authblk}

\title{Origin of Rapid Delithiation In Secondary Particles Of \nca{} and \nmc{} Cathodes}

\author[1,2,3]{Mark Wolfman}
\author[1]{Brian M.\ May}
\author[4]{Vishwas Goel}
\author[4]{Sicen Du}
\author[5,6]{Young-Sang Yu}
\author[7]{Nicholas V.\ Faenza}
\author[7]{Nathalie Pereira}
\author[8]{Antonin Grenier}
\author[3]{Kamila M.\ Wiaderek}
\author[3]{Ruqing Xu}
\author[9]{Jiajun Wang}
\author[8]{Karena W.\ Chapman}
\author[7]{Glenn G. Amatucci}
\author[4]{Katsuyo Thornton}
\author[1]{Jordi Cabana\thanks{Corresponding author: jcabana@uic.edu}}

\affil[1]{Department of Chemistry, University of Illinois at Chicago}
\affil[2]{Chemical Sciences and Engineering Division, Argonne National Laboratory}
\affil[3]{X-ray Science Division, Advanced Photon Source, Argonne National Laboratory}
\affil[4]{Department of Materials Science and Engineering, University of Michigan}
\affil[5]{Department of Physics, Chungbuk National University}
\affil[6]{Advanced Light Source, Lawrence Berkeley National Laboratory}
\affil[7]{Energy Storage Research Group, Department of Materials Science and Engineering, Rutgers, The State University of New Jersey}
\affil[8]{Department of Chemistry, Stony Brook University}
\affil[9]{Harbin Institute of Technology}

\date{}

\newcommand{\dedt}{\frac{\textrm{d}E}{\textrm{d}t}}

%% Custom units
\DeclareSIUnit\Molar{\textsc{m}}
\DeclareSIUnit\molar{\mole\per\cubic\deci\metre}


\newcommand{\nca}[1]{\ce{Li_{#1}Ni_{0.8}Co_{0.15}Al_{0.05}O_2}}
\DeclareRobustCommand{\nmc}[2][]{%
    \ifstrempty{#1}{%
        \ce{Li_{#2}Ni_{y}Mn_{z}Co_{1-y-z}O2}}{}%
    \ifstrequal{#1}{333}{%
        \ce{Li_{#2}Ni_{1/3}Mn_{1/3}Co_{1/3}O2}}{}%
    \ifstrequal{#1}{532}{%
        \ce{Li_{#2}Ni_{0.5}Mn_{0.3}Co_{0.2}O2}}{}%
}


\makeatletter
\newcommand*{\addFileDependency}[1]{% argument=file name and extension
  \typeout{(#1)}
  \@addtofilelist{#1}
  \IfFileExists{#1}{}{\typeout{No file #1.}}
}
\makeatother

\newcommand*{\myexternaldocument}[2][]{%
  \ifstrempty{#1}{%
    \externaldocument{#2}%
  }{%
    \externaldocument[#1]{#2}%
  }%
  \addFileDependency{#2.tex}%
  \addFileDependency{#2.aux}%
}

\newacronym{txm}{TXM}{transmission X-ray microscopy}
\newacronym{xas}{XAS}{X-ray absorbance spectroscopy}


\begin{document}

\maketitle

%%%%%%%%%%%%%%%%%%%%%%%
\section{Introduction}
%%%%%%%%%%%%%%%%%%%%%%%

\blindtext

%%%%%%%%%%%%%%%%%%
\section{Results}
%%%%%%%%%%%%%%%%%%

\begin{figure}
  \includegraphics{figures/NCA_xrd.png}
  \caption{Operando \gls{xrd} of \nca{} secondary particle
    agglomerates during first charge. \todo{Write a more detailed
      caption.}}
\end{figure}

\begin{figure}
  \includegraphics{figures/nca_txm.pdf}
  \caption{Operando \gls{txm} of \nca{} during first charge. (a) Mean
    optical depth frame of \nca{} particles. (b) Applied potential to
    operando cell during galvanostatic charging at \todo{what current,
      mA/g}. (c) Normalized spectra from \emph{ex-situ}
    ensemble-average \gls{xas}. (d) State-of-charge determined by
    whiteline position relative to overall state of charge in (c) for
    individual particles of \nca{}. Error bars represent one standard
    deviation over pixels within the given particle.}
\end{figure}

\begin{figure}
  \includegraphics{figures/nmc_txm.pdf}
  \caption{Operando \gls{txm} of \nmc[333]{} and \nmc[532]{} during
    first charge. (a) Mean optical depth frame of \nmc[333]{}
    particles. (b,c) Median optical depth spectra of active material
    during charging for (b) \nmc[333]{} and (c) \nmc[532]{}. (d,e)
    Changes in median whiteline energies relative to start of charging
    cycle for individual particles of (d) \nmc[333] and (e)
    \nmc[532]{}.}
\end{figure}


%%%%%%%%%%%%%%%%%%%%%
\section{Discussion}
%%%%%%%%%%%%%%%%%%%%%

\Blindtext

%%%%%%%%%%%%%%%%%%%%%%%%%%%%%%%%
\section{Materials and Methods}
%%%%%%%%%%%%%%%%%%%%%%%%%%%%%%%%

\blindtext

%%%%%%%%%%%%%%%%%%%%%
\section{Conclusion}
%%%%%%%%%%%%%%%%%%%%%

\blindtext

\end{document}

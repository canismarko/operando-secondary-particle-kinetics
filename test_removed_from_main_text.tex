%%modeling section

The results shown here are also entirely
consistent with those from Park et al.\ and provide further insight
into this phenomenon by way of operando \gls{uxrd} and \gls{txm}--\gls{xas} used to build detailed models of the delithiation kinetics,
as well as including \nca{} in the cathode materials exhibiting this behavior. Furthermore, the rapid and stochastic delithiation of secondary particles is consistent with the operando visible-light microscopy results in Sharma et al\cite{zhao2022}. However, their models showed that this behavior was due to electrical conductivity in the activate material along with connectivity to the carbon matrix. We ruled out the later explanation since the stochastic behavior was observed even when using electrodes prepared with a high content of conductive carbon (\SI{60}{\percent} w/w).

Several studies instead attributed the rapid delithiation to limited diffusion of \ce{Li^+} in the cathode active material\cite{rao2021, wang2020-6}. Their models and those
described above predict spatial
heterogeneity within secondary particles in the diffusion limited case\cite{wang2020-6}, which is
not observed for polycrystalline secondary particles, neither in this
work nor that of other authors\cite{chueh2021, zhao2022}. This difference between
polycrystalline and single crystal cathode behavior implies that
\ce{Li^+} diffusion may be the limiting factor in the case of large
single crystal particles, but that poly-crystalline secondary
particles are instead limited by charge-transfer kinetics.
